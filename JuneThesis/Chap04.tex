% Add an analysis section

The work performed in this thesis seeks to add to the growing body of knowledge surrounding Nuclear Renewable Hybrid Energy Systems as a solution for nuclear power plants and renewable energy sources to work in tandem to decarbonize both the electric and industrial sectors in an economically profitable way. There is much more work to be done both on this particular example of coupling a water purification system to a Palo Verde reactor as well as more generally in determining the appropriate application of NRHESs. Some of the future work suggested includes:

\begin{itemize}
\item Use a single industrial process to analyze thermal versus electrical coupling in a NRHES.
\item Include flexibility in a more dynamic manner instead of a qualitative assessment of the differences between MSF and RO desalination.
\item Quantify the quality of the product produced with the two different approaches.  The RO system would produce high quality water initially while a MSF system would require some startup time, a feature which was not quantified in this analysis.
\item Include a financial value for flexibility and the resilience provided to a nuclear power plant that produces its own water.
\item Further develop the Specific Exergy Revenue thermoeconomic approach to include multiple products.
\item Perform a sensitivity analysis of the exergy economics analysis to the price of water and electricity.
\item Compare the nuclear exergy analysis with a coal plant and concentrated solar analysis to see how the thermodynamic properties of various heat sources compare.
\item Model a battery system, such as compressed air, instead of the water purification system to compare the thermodynamic and economic benefits of both.
\item Perform an economic assessment of the revenue lost if nuclear plants were asked to curtail, then slowly ramp back up, as opposed to sending electricity to an industrial process.
\item Conduct an analysis of a combination of thermal heat and electric power using some of the waste heat from the reactor to pre-heat the water entering the reverse osmosis system.
\item Use more realistic pumps for the RO system.  The three pumps included in this analysis are off-the-shelf options for RO systems.  The RO system would likely be tailored to the use case for Palo Verde.
\item Use a more precise power cycle and thoroughly determine the thermalhydraulic feedbacks included with thermal coupling.
\item Include capitol and maintenance costs for the water purification systems.
\item Focus on taking individual turbines off of the system as opposed to specific loads.  Taking each of the four turbines off of the system and sending that heat to a water purification system may be a better management technique for the power plant.
\item Include a more thorough analysis of how bypassing heat exchangers or using reheaters in the power cycle would affect the overall exergetic efficiency.
\item Include a regulatory review and investigation to analyze the likely regulatory complexities of thermal versus electrical coupling of industrial processes.
\end{itemize}

This research ended up displaying the benefits of the two different water purification technologies more so than the benefits of thermally coupling an industrial process to a nuclear power plant in a NRHES arrangement.  In general, the technologies used for a thermally coupled versus an electrically coupled system will differ some. In order to quantify and describe the benefits of thermally coupling in a NRHES would require choosing an industrial process which strictly needs heat. Then the systems could be compared based on one of the systems using the heat from the reactor and the other using heat converted from electricity in the reactor.

Some of the future work to be done on applying AHP to decision making in NRHES include ensuring that the characteristics included in the analysis are valuable relative measures as opposed to strictly meeting a standard. The safety and ability to fluctuate characteristics should meet a certain clearly defined standard.  A comparative approach was inappropriate for characteristics that either meet a standard or do not.

\section{Summary Remarks}

 There were multiple approaches taken in this thesis to analyze how to progress in the modeling and development of NRHESs. The initial literature review discussed the progression of typically smaller traditional \ac{hes}s as well as the development up to this point for \ac{nrhes}s.  The literature review concluded that future research on modeling NRHESs should compliment the ongoing work modeling NRHESs using the Modelica language and RAVEN.  While the ongoing modeling focuses on questions surrounding how to optimize a NRHES based on minimizing cost, Chapter \ref{TvsE} focused on comparing the thermodynamic and economic benefits of electric and thermal coupling of two different water purification systems. Chapter \ref{Risk} focused on applying two risk assessment techniques, \ac{pha} and \ac{ahp}, to a NRHES.  The chapter discussed the benefits of applying risk assessment techniques early to large capital intensive projects to minimize financial and safety risks.  The chapter also discussed an applied expert AHP survey.  While the fuzzy logic applied to the survey suggested that desalination was the best industrial process to include in a NRHES, that conclusion primarily came from the large emphasis placed on safety.  Future AHP applications should focus on characteristics of the system which do not need to meet clearly defined standards, but are of more relative importance.  The safety and flexibility characteristics used in the AHP have clearly defined legal standards which are either met or not met, not judged on their relative merit.  Chapter \ref{TvsE} focused on comparing the thermodynamic and economic benefits of coupling a water purification system to Palo Verde Generating Station. It found that in general a reverse osmosis water purification electrically coupled to a nuclear power plant has greater revenue as well as better exergetic economics, as well as being simpler to operate.

 In conclusion, Nuclear Renewable Hybrid Energy systems propose a possible solution allowing nuclear power plants to flexibly operate along with grid demand and renewable generation.  When the price of electricity is very low, there is an economic benefit to selling the heat or electricity to a water distillation system for the Palo Verde Nuclear Power Plant. Also, as the price of water continues to increase in value, there is an economic benefit to producing more water. At current average prices for electricity, there is no revenue advantage to generating water as well as electricity. When the price of electricity is less than \$32/MWh, it does make sense to generate some water with the reverse osmosis systems. The benefits from the flexibility as well as the stability of generating sufficient water for the power plant have not been included in the economic assessment. These characteristics do have a value which needs to be included in future assessments. While there is a lot of work to be done analyzing hybrid energy systems, there is some promise that, depending on the price for electricity and water, they could have a valuable role in allowing nuclear power plants adapt to a changing grid.
